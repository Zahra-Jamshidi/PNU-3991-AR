\documentclass{book}
\begin{document}
\begin{flushright}
\hspace*{0.5cm}
\textbf{35}
\end{flushright}
\vspace*{0.7}
the advantages of each method,providing different perspectives of out world (Eisner,1981;Firestone,1987;Glesne & Peshkin,1992). For example, the types of methods most commonly associated with quantitative research are experimental studies and questionnaires; the types of methods most commonly associated with qualitative research are ethnographies, grounded theory investigations, case studies, interviews, and phenomenological studies (Creswell,1994).while these methods are commonly associated with one paradigm,there are occasions when qualitative researchers will use questionnaires and quantitative researchers will use interviews (Glesne & Peshkin); this results in a mixed method study. Some research theorists argue that qualitative and quantitative approaches are incompatible (Lincoln & Guba, 1985; Schwandt,1989). This position is based on the argument that each research paradigm rests on different assumptions about the nature of the world, requiring different instruments and procedures for investigation. Glesne and Peshkin further explain that most researchers have a tendency to adhere to the methodology that is most consistent with their socialized worldview:
we are told that the research problem should define whether one chooses a qualitative approach or a quantitative one. This, however, is not how we believe research necessarily is done or even how it should be done. To the country, we are attracted to and shape the research problems that match our personal world view of seeing and under-standing the world.(p.9)

Thus, it is important for e-researchers to be self-aware to the extent that they can articulate their own personal worldview. If you are a beginning researcher, our advice is to real many different research studies that use a wide variety of methodologies. From this overview, certain approaches will strike you as more effective in addressing the questions and providing answers that are meaningful to you. This worldview should guid your selection of an appropriate question, paradigm, and methodology to use in your e-research study.


\subsubsection{MATCHING THE LANGUAGE WITH THE METHOD}


As mentioned in the previous section, an essential aspect of the problem statement and research question is the expression of the assumptions of the research paradigm through language that is consistent with the research method. When the Net is used in the research process, care should also be given to ensure that the language and method correspond.
\\\hspace*{0.5cm} In qualitative research, the research quesstion is often presented in the form of a grand tour question, often followed by subquestions (Creswell,1994). A grand tour question, according to Werner and Schoepfle (1987), is constructed through the use of nondirective wording, which ensures unlimited inquiry. For example, words that convey the language of an emerging design of qualitative research will include nondirectional language, such as discover, explain, seek to understand, explore, and describe the experiences. Generally, there is a single focus and a specific research situation, without reference to the literature or theory. This is not to say a review of the literature on
\end{document} 