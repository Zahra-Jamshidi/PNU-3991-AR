\documentclass[12pt]{beamer}
\usepackage{xcolor}
\usefonttheme{serif}
\usepackage{tikz}
\usetheme{Darmstadt}
\author{pages 34-36}
\title{E-research methods,strategies,and issuse}
\begin{document}
\frame{\maketitle}
\begin{frame}
qualitative and quantitative research-the results of which are analyzed together to triangulate or provide converging views from different perspectives on a problem. Action research involves a partunership between the researcherand those immersed in a situation; if focuses on solutions to real problems in daily life. In practice, all of these methods can be broadly classified under one of two educational paradigms-quantitative or qualitative. We will not attempt to recap the development or the often-heated arguments between proponents of these two paradigms. However, it is important to discuss their differences and to note, where applicable. how the Internet changes these methods of inquiry.
\end{frame}
\begin{frame}
 To understand the differences between the two most common approaches to research (quantitative and qualitative), the literature has divided the two paradigms on several dimensions (Glesne and Peshkin, 1992; Guba and Lincoln, 1988; McCracken, 1988).
 while these dimensions are often not fully applicable to actual research as patton (1988) notes, they highlight the contrasts of each paradigm. The most fundamantal difference in each paradigm is the assumption about the nature of the problem.
\end{frame}
\begin{frame}
 In particular, the method e-researchers choose often reveals what they value as knowledge and, perhaps more importantly, what their perspective of the nature of reality is, for example, quantitative methods are based on an assumption that the world is comprised of observable and measurable facts.
  With respect to e-research, this involves using the Internet as a data-gathering tool to find cause and effect through an objective, formal, and deductive design, which aims to be contex and value free. Eresearchers need to guard against their own biases by remaining as neutral and as independent as possible from their research.
\end{frame}
\begin{frame}
It is also important to acheive accuracy and reliability through established statistical validity and reliability practices (Borg and Gall, 1989; fraenkel and Wallen, 2000). Quantitative researchers strive for transparency and objectivty, thus allowing their results to generalize to other places and people. currently, many e-researchers using quantitative studies assume that the use of the Net for data collection is no different, with respect to reliability and validity, from more traditional forms of data collection. However, as we will see in the following discussion, this is not necessarily a correct assumption.
In contrast, qualitative methodes assume multiple realities, which are subjective, value-laden, complex, and contex-bound. Accuracy is achieved through credibility, transferability, confirmability, and dependability (Guba, 1981).
\end{frame}
 \begin{frame}
What results is the discovery of patterns and the development of theories that expand our understanding through narratives that "exploit the power of form to inform" (Eisner, 1981, p.7).
 The qualititative e-researcher interacts with the research using an in-depth inductive process and an emerging design that is identified during the research process. With respect to the Net,the qualitative e-researcher generally views it as more than simply a tool for data gathering. In particular, the e-research e-researcher's and subject's assumptions and values in relation to the role of the Net will be part of the research process. Thus, the Net and the research results are indivisible, in that the e-researcher's biases and beliefs about the Net will influence the research process, and hence, should be reflected on and addressed throughout the study.
\end{frame}
 \begin{frame}
  The different assumptions of each research approach will not only influence the methodogical approach selected, but also the purpose of the research and the role of the e-researcher. Some researchers argue that integrating diverse approaches draws on the advantages of each method,providing different perspectives of out world (Eisner,1981;Firestone,1987;Glesne and Peshkin,1992).
 For example, the types of methods most commonly associated with quantitative research are experimental studies and questionnaires; the types of methods most commonly associated with qualitative research are ethnographies, grounded theory investigations, case studies, interviews, and phenomenological studies (Creswell,1994).
\end{frame}
 \begin{frame}
while these methods are commonly associated with one paradigm,there are occasions when qualitative researchers will use questionnaires and quantitative researchers will use interviews (Glesne and Peshkin); this results in a mixed method study.
 Some research theorists argue that qualitative and quantitative approaches are incompatible (Lincoln and Guba, 1985; Schwandt,1989). This position is based on the argument that each research paradigm rests on different assumptions about the nature of the world, requiring different instruments and procedures for investigation.
\end{frame}
\begin{frame}
 Glesne and Peshkin further explain that most researchers have a tendency to adhere to the methodology that is most consistent with their socialized worldview:
we are told that the research problem should define whether one chooses a qualitative approach or a quantitative one.
 This, however, is not how we believe research necessarily is done or even how it should be done.
To the country, we are attracted to and shape the research problems that match our personal world view of seeing and under-standing the world.(p.9)
\end{frame}
\begin{frame}
the advantages of each method,providing different perspectives of out world (Eisner,1981;Firestone,1987;Glesne and Peshkin,1992). For example, the types of methods most commonly associated with quantitative research are experimental studies and questionnaires; the types of methods most commonly associated with qualitative research are ethnographies, grounded theory investigations, case studies, interviews, and phenomenological studies (Creswell,1994).
while these methods are commonly associated with one paradigm,there are occasions when qualitative researchers will use questionnaires and quantitative researchers will use interviews (Glesne and Peshkin); this results in a mixed method study.
\end{frame}
\begin{frame}
 Some research theorists argue that qualitative and quantitative approaches are incompatible (Lincoln and Guba, 1985; Schwandt,1989). This position is based on the argument that each research paradigm rests on different assumptions about the nature of the world, requiring different instruments and procedures for investigation. Glesne and Peshkin further explain that most researchers have a tendency to adhere to the methodology that is most consistent with their socialized worldview:
we are told that the research problem should define whether one chooses a qualitative approach or a quantitative one.
\end{frame}
 \begin{frame}
 This, however, is not how we believe research necessarily is done or even how it should be done. To the country, we are attracted to and shape the research problems that match our personal world view of seeing and under-standing the world.(p.9)


MATCHING THE LANGUAGE WITH THE METHOD


 As mentioned in the previous section, an essential aspect of the problem statement and research question is the expression of the assumptions of the research paradigm through language that is consistent with the research method. When the Net is used in the research process, care should also be given to ensure that the language and method correspond.
\end{frame}
 \begin{frame}
  In qualitative research, the research quesstion is often presented in the form of a grand tour question, often followed by subquestions (Creswell,1994). A grand tour question, according to Werner and Schoepfle (1987), is constructed through the use of nondirective wording, which ensures unlimited inquiry. For example, words that convey the language of an emerging design of qualitative research will include nondirectional language, such as discover, explain, seek to understand, explore, and describe the experiences.
 Generally, there is a single focus and a specific research situation, without reference to the literature or theory.
\end{frame}
\begin{frame}
 This is not to say a review of the literature on related theories is not part the process of conducting qualitative research; rather, as the inttended outcome is often theory development, reference to a particular theory is typically not made in the problem statement or the e-research question.
  Grand tour questions in e-research are generally--thought not always--followed by five to twelve subquestions that narrow the focus of the study and describe the use of the Net in the research process, but still permit for an emerging research design (Creswell,1994;Miles and Huberman,1984).
\end{frame}
\begin{frame}
The following are examples of grand tour questions used in different research questions.

Example 1. Grounded theory, open coding : What are the categories of discussion that emerge from Web-based threaded conferencing interactions between North American and European cultures?

Example 2. Phenomenology: What is it like for children who are geographically isolated to participate in learning activities delivered over the Internet?

Example 3. Ethnography: How is social presence established in Web-based learning activities?
 \end{frame}
 \begin{frame}
Alternatively, in quantitative research, the research question is often presented with a narrow focus--sometimes as a statement rather than a question--followed by restating the purpose of the study (Creswell,1994). Generally, these restatements are in the form of research questions (or objectives) if it is a survey project and hypotheses if it is an experimental project. Hypotheses are declarative statements that make a tentative proposition about relationships, but are phrased as questions (Borg and Gall,1989). Quantitative problem statements are typically constructed through the use of directive wording, which ensures the research propositions are testable.
\end{frame}
\begin{frame}
 For example, words that convey the language of an experimental design will include directional language, such as affect, influence,compare, determine, lead to,cause, and relate. The research questions or hypotheses are usually derived from a theory that provides a framework for the study.
 As one goal of quantitative research is to increase our understanding is through the use of theory (Vierra, Pollock, and Golez,1998). In a sense, theories are proposed explanations that researchers test for rejection or acceptance.
\end{frame}
\begin{frame}
However, as Vierra et al. note, rarely are theories rejected or acceptted; rather, they are most often revised. Hence, researchers generally make reference to their work as theory building, which reflects the ongoing process of testing, revising, and retesting. In quantitative studies, researchers test theories through a deductive methodological process--often classified as either a causal comparative method or a descriptive study. The causal comparative method (often associated with grounded theory methodology) is primarily concerned with discovering causal relationships and involves a comparision between two or more groups in terms of a dependent variable or as a relationship of two or more independent and dependent variables.
\end{frame}
\begin{frame}
A descriptive study is primarily concerned with discovering "what is" and involves describing responses based on the independent or dependent variables. Observational and survey methods are frequently used to collect
 \end{frame}
 \end{document}
