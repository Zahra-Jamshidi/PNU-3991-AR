\documentclass{book}
\begin{document}
\begin{flushright}
\hspace*{0.5cm}
\textbf{35}
\end{flushright}
\vspace*{0.7}
related theories is not part the process of conducting qualitative research; rather, as the inttended outcome is often theory development, reference to a particular theory is typically not made in the problem statement or the e-research question.
\\\hspace*{0.5cm} Grand tour questions in e-research are generally--thought not always--followed by five to twelve subquestions that narrow the focus of the study and describe the use of the Net in the research process, but still permit for an emerging research design (Creswell,1994;Miles & Huberman,1984). The following are examples of grand tour questions used in different research questions.
\begin{quote}
Example 1. Grounded theory, open coding : What are the categories of discussion that emerge from Web-based threaded conferencing interactions between North American and European cultures?

Example 2. Phenomenology: What is it like for children who are geographically isolated to participate in learning activities delivered over the Internet?

Example 3. Ethnography: How is social presence established in Web-based learning activities?
\end{quote}
Alternatively, in quantitative research, the research question is often presented with a narrow focus--sometimes as a statement rather than a question--followed by restating the purpose of the study (Creswell,1994). Generally, these restatements are in the form of research questions (or objectives) if it is a survey project and hypotheses if it is an experimental project. Hypotheses are declarative statements that make a tentative proposition about relationships, but are phrased as questions (Borg & Gall,1989). Quantitative problem statements are typically constructed through the use of directive wording, which ensures the research propositions are testable. For example, words that convey the language of an experimental design will include directional language, such as affect, influence,compare, determine, lead to,cause, and relate. The research questions or hypotheses are usually derived from a theory that provides a framework for the study.
\\\hspace*{0.5} As one goal of quantitative research is to increase our understanding is through the use of theory (Vierra, Pollock, & Golez,1998). In a sense, theories are proposed explanations that researchers test for rejection or acceptance.However, as Vierra et al. note, rarely are theories rejected or acceptted; rather, they are most often revised. Hence, researchers generally make reference to their work as theory building, which reflects the ongoing process of testing, revising, and retesting. In quantitative studies, researchers test theories through a deductive methodological process--often classified as either a causal comparative method or a descriptive study. The causal comparative method (often associated with grounded theory methodology) is primarily concerned with discovering causal relationships and involves a comparision between two or more groups in terms of a dependent variable or as a relationship of two or more independent and dependent variables. A descriptive study is primarily concerned with discovering "what is" and involves describing responses based on the independent or dependent variables. Observational and survey methods are frequently used to collect
\end{document} 