\documentclass{book}
\begin{document}
\begin{flushright}
\hspace*{0.5cm}
\textbf{34}
\end{flushright}
\vspace*{0.7}
qualitative and quantitative research-the results of which are analyzed together to triangulate or provide converging views from different perspectives on a problem. Action research involves a partunership between the researcherand those immersed in a situation; if focuses on solutions to real problems in daily life. In practice, all of these methods can be broadly classified under one of two educational paradigms-quantitative or qualitative. We will not attempt to recap the development or the often-heated arguments between proponents of these two paradigms. However, it is important to discuss their differences and to note, where applicable. how the Internet changes these methods of inquiry.
\\\hspace*{0.5cm} To understand the differences between the two most common approaches to research (quantitative and qualitative), the literature has divided the two paradigms on several dimensions (Glesne and Peshkin, 1992; Guba and Lincoln, 1988; McCracken, 1988). while these dimensions are often not fully applicable to actual research as patton (1988) notes, they highlight the contrasts of each paradigm. The most fundamantal difference in each paradigm is the assumption about the nature of the problem. In particular, the method e-researchers choose often reveals what they value as knowledge and, perhaps more importantly, what their perspective of the nature of reality is, for example, quantitative methods are based on an assumption that the world is comprised of observable and measurable facts. With respect to e-research, this involves using the Internet as a data-gathering tool to find cause and effect through an objective, formal, and deductive design, which aims to be contex and value free. Eresearchers need to guard against their own biases by remaining as neutral and as independent as possible from their research. It is also important to acheive accuracy and reliability through established statistical validity and reliability practices (Borg & Gall, 1989; fraenkel & Wallen, 2000). Quantitative researchers strive for transparency and objectivty, thus allowing their results to generalize to other places and people. currently, many e-researchers using quantitative studies assume that the use of the Net for data collection is no different, with respect to reliability and validity, from more traditional forms of data collection. However, as we will see in the following discussion, this is not necessarily a correct assumption.
In contrast, qualitative methodes assume multiple realities, which are subjective, value-laden, complex, and contex-bound. Accuracy is achieved through credibility, transferability, confirmability, and dependability (Guba, 1981). What results is the discovery of patterns and the development of theories that expand our understanding through narratives that "exploit the power of form to inform" (Eisner, 1981, p.7). The qualititative e-researcher interacts with the research using an in-depth inductive process and an emerging design that is identified during the research process. With respect to the Net,the qualitative e-researcher generally views it as more than simply a tool for data gathering. In particular, the e-research e-researcher's and subject's assumptions and values in relation to the role of the Net will be part of the research process. Thus, the Net and the research results are indivisible, in that the e-researcher's biases and beliefs about the Net will influence the research process, and hence, should be reflected on and addressed throughout the study.
\\\hspace*{0.5cm} The different assumptions of each research approach will not only influence the methodogical approach selected, but also the purpose of the research and the role of the e-researcher. Some researchers argue that integrating diverse approaches draws on
\end{document} 